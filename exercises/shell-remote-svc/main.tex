\PassOptionsToPackage{dvipsnames, svgnames}{xcolor}
\documentclass[a4paper, 12pt]{article}
\usepackage[english]{babel}
\usepackage[utf8]{inputenc} 
\usepackage[T1]{fontenc}
\usepackage[margin=2cm]{geometry}
\usepackage{etoolbox}
\usepackage[hidelinks]{hyperref}
\usepackage{listings}
\usepackage{tabularx}
\usepackage{tcolorbox}
\usepackage{times}

% Define background color
\colorlet{LightGray}{Gray!15!}

% Syntax coloration (better than listings but requires --shell-escape=1
\usepackage{minted}
\usemintedstyle{borland}

\title{Exercise 1: Hands on Pelix Shell and Remote Services}
\author{Thomas Calmant}
\date{}

% Counter for questions
\newcounter{question}
\newcommand\Question{%
  \stepcounter{question}%
  \paragraph{\hspace{2ex} Question \thequestion.}
}

% Better lists
\let\tempone\itemize
\let\temptwo\enditemize
\renewenvironment{itemize}{\tempone\addtolength{\itemsep}{-.5em}}{\temptwo}

% Note block
\tcbuselibrary{skins,breakable}
\newenvironment{noteblock}[1]{%
    \tcolorbox[beamer,%
    noparskip,breakable,
    colback=LightGreen,colframe=DarkGreen,%
    colbacklower=LimeGreen!75!LightGreen,%
    title=\textbf{#1}]}%
    {\endtcolorbox}

\begin{document}
\maketitle

\section{Setup your environment}

\begin{enumerate}
\item Ensure you have Python 3.3+ or 2.7 installed
\item Install iPOPO
\begin{minted}{bash}
  pip install --user iPOPO
\end{minted}
\item Run a Pelix shell with the following command
\begin{minted}{bash}
  python -m pelix.shell
\end{minted}
\end{enumerate}

\section{Play with the shell}

\subsection{Basic commands}

\Question In the shell, try the following commands and explain what they do:
\begin{itemize}
\item \texttt{bl}
\item \texttt{sl}
\item \texttt{bd 1}
\item \texttt{sl 1}
\item \texttt{threads}
\item \texttt{update 2}
\end{itemize}

\subsection{New command}

\begin{listing}[ht]
\inputminted[fontsize=\footnotesize, frame=lines, framesep=2mm, baselinestretch=1.2, bgcolor=LightGray, linenos]{python}{../shell_hello.py}
\caption{\texttt{shell\_hello.py}: Registers a \textit{sample.hello} shell command}
\label{code:shell_hello}
\end{listing}

Listing \ref{code:shell_hello} is the code of a bundle which provides an \textit{hello} command in \textit{sample} name space.

\begin{enumerate}
\item Copy and paste the code in a file called \texttt{shell\_hello.py}
\item Install the bundle in Pelix, using \texttt{install shell\_hello}
\item Start it, using the \texttt{start} command
\item Verify if the \texttt{hello} command is registered, using \texttt{help}
\item Call the command
\end{enumerate}

\Question The \texttt{hello} command supports only one argument. Update it so that it accepts a variable number of arguments, like: \texttt{hello Wonderful World}.

\section{The \texttt{hostname} service}

After having imported the \href{https://docs.python.org/3/library/socket.html}{\texttt{socket}} module, it is possible to the host name of your machine using the \href{https://docs.python.org/3/library/socket.html#socket.gethostname}{\texttt{gethostname()}} method.

\Question Create a new bundle, \textit{i.e.} a new Python module, that defines a new component factory with the following constrains:
\begin{itemize}
\item The components of this factory must provide a service with the \texttt{ex1.hostname} specification;
\item A component must instantiated as soon as the factory is loaded;
\item The components must provide a \texttt{get\_hostname()}, which returns the host name of the local machine.
\end{itemize}

\Question Install the new bundle and verify that:
\begin{itemize}
\item a component has been instantiated;
\item a service implementing \texttt{ex1.hostname} is registered.
\end{itemize}

\Question Update the \texttt{shell\_hello} bundle to register an \texttt{hostname} shell command.
This command must use the \texttt{ex1.hostname} service, provided by the component instantiated with the previous command.

\section{Remote Services}

\Question Update the \texttt{hostname} shell command implementation so that it prints the result of \texttt{get\_hostname()} of all registered \texttt{ex1.hostname} services.
You should use the \texttt{aggregate} property of the \texttt{@Requires} decorator.

\subsection{Export Property}

To be exported, a service must be associated to an export property.
The most commonly used is \texttt{service.exported.interfaces}, set to either \texttt{*} (export all service interfaces) or to a list of interfaces allowed to be exported.

\Question Add the export property to the \texttt{ex1.hostname} service, using the \texttt{@Property} decorator.

\subsection{Play with Remote Services}

\begin{noteblock}{Check the access to your peers before trying to call them...}
Ensure that you are on the same network as your peers before starting remote services components.
Also ensure that there are no firewall rules on the router and peers which would reject the multicast or the TCP packets.

\vspace{1em}
If you're alone, simply run a second framework in parallel of the current one, with the same bundles installed.
\end{noteblock}

\vspace{1em}

Remote Services are handled by a set of bundles which are not installed by default in a Pelix framework.
You'll have to install:

\begin{center}
\begin{tabularx}{.95\textwidth}{r X}
\texttt{\footnotesize pelix.remote.dispatcher} & The dispatcher, handling the exported services \\
\hline
\texttt{\footnotesize pelix.remote.registry} & The registry, handling the imported services \\
\hline
\texttt{\footnotesize pelix.remote.discovery.multicast} & A discovery service, based on multicast datagrams \\
\hline
\texttt{\footnotesize pelix.remote.json\_rpc} & A transport service to call remote methods, \par based on JSON-RPC \\
\hline
\texttt{\footnotesize pelix.http.basic} & The Pelix HTTP Service, required by JSON-RPC \\
\end{tabularx}
\end{center}

When those bundles are installed, it is necessary to instantiate some components of the factories they provide.
During this step, it is possible to configure the addresses and ports used by the HTTP service and the discovery services.

Listing \ref{code:remote_svc_install} is a script file that can be executed in the Pelix shell using the \texttt{run} command.
It installs the remote services bundles and instantiates the required components.

\begin{listing}[!hb]
\inputminted[fontsize=\footnotesize, frame=lines, framesep=2mm, baselinestretch=1.2, bgcolor=LightGray, linenos]{python}{../remote_svc.txt}
\caption{\texttt{remote\_svc.txt}: A Pelix Shell script that installs remote services bundles and instantiates the required components}
\label{code:remote_svc_install}
\end{listing}

\Question Now that the remote services components are up and running, check with \texttt{sl} if you see some imported services:
\begin{itemize}
\item How can you tell that a service has been imported?
\item Which bundle provides the imported services?
\end{itemize}

\Question Try your \texttt{hostname} shell command, what happens?

\section{Useful links}

Here are some links to the iPOPO project which can help you during this exercise:
\begin{itemize}
\item List of the commands available in the Pelix Shell\\
\begin{footnotesize}
\url{https://ipopo.coderxpress.net/wiki/doku.php?id=ipopo:refcards:shell}
\end{footnotesize}

\item How to use the shell and how to write a command provider:\\
\begin{footnotesize}
\url{https://ipopo.coderxpress.net/wiki/doku.php?id=ipopo:tutorials:shell} 
\end{footnotesize}

\item How to export a service\\
\begin{footnotesize}
\url{https://ipopo.coderxpress.net/wiki/doku.php?id=ipopo:tutorials:remote_svc} 
\end{footnotesize}
\end{itemize}
\end{document}