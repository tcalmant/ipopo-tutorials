\subsection{iPOJO Model}

\begin{frame}{Component Life Cycle}
\centering
\vspace{1em}

\tikzset{
    %Define standard arrow tip
    >=stealth',
    %Define style for boxes
    punkt/.style={
           rectangle,
           rounded corners,
           draw=black, very thick,
           fill=yellow!30,
           text width=7em,
           minimum height=2em,
           text centered},
    % Define arrow style
    pil/.style={
           ->,
           thick,
           shorten <=2pt,
           shorten >=2pt,}
}

\begin{tikzpicture}[node distance=2em, auto]
\node[circle, draw, very thick, fill=yellow!30] (init) {};
\node[punkt, below=of init] (instantiated) {INSTANTIATED};
\node[punkt, right=8em of instantiated] (validating) {VALIDATING};
\node[punkt, above=of validating] (erroneous) {ERRONEOUS};
\node[punkt, below=of validating] (valid) {VALID};
\node[punkt, below=of valid] (invalidating) {INVALIDATING};
\node[punkt, below=of instantiated] (deleted) {DELETED};
\node[circle, draw, very thick, fill=black, below=of deleted] (final) {};

\draw[pil] (init.south) to node[auto, swap] {Instantiate} (instantiated.north);
\draw[pil] (instantiated.east) to node[auto, swap] {Validate} (validating.west);
\draw[pil] (validating.north) to node[auto, swap] {<exception>} (erroneous.south);
\draw[pil] (erroneous.west) to node[auto, swap] {Retry}(instantiated.north east);
\draw[pil] (validating.south) to node[auto] {} (valid.north);
\draw[pil] (valid.south) to node[auto] {Invalidate} (invalidating.north);
\draw[pil] (invalidating.west) to node[auto] {} (instantiated.east);
\draw[pil] (instantiated.south) to node[auto, swap] {Remove} (deleted.north);
\draw[pil] (deleted.south) to node[auto] {} (final.north);
\end{tikzpicture}

\begin{small}
\begin{exampleblock}{Note}
Similar life cycle as in iPOJO, with an additional \texttt{ERRONEOUS} state if the validation fails.
\end{exampleblock}
\end{small}
\end{frame}


\subsection{Code Snippets}

\begin{frame}[fragile]{Hello World sample}
\begin{block}{Provider component}
\begin{minted}{python}
@ComponentFactory("my-provider")
@Provides(MyInterface)
@Requires("logger", pelix.misc.LogService)
@Instantiate("my-provider-instance")
class MyImplementation(MyInterface):
    # Here, all injected members should be declared
    # in order to give them the right type
    logger: pelix.misc.LogService

    def hello(self, name):
        # This method is available to all
        # "my-interface" consumers
        print("Hello,", name, "!")
        self.logger.log(
            logging.LOG_DEBUG, f"Said hello to {name}"
        )
\end{minted}
\end{block}
\end{frame}

\begin{frame}[fragile]{Hello World sample}
\begin{block}{Consumer component}
\begin{minted}{python}
@ComponentFactory(name="my-consumer")
@Requires("service", MyInterface)
@Property("name", "my.name", "World")
@Instantiate("my-consumer-instance")
class MyConsumer:
    service: MyInterface
    name: str | None

    @Validate
    def validated(self):
        print("Consumer validated!")
        self.service.hello(name)

    @Invalidate
    def invalidated(self):
        print("Consumer invalidated!")
        self.service.hello("the end")
\end{minted}
\end{block}
\end{frame}


\begin{frame}[fragile]{Hello World sample}
\begin{block}{Result}
\begin{minted}{bash}
** Pelix Shell prompt **
$ install hello.provider
Bundle ID: 12
$ start 12
Starting bundle 12 (hello.provider)...
$ install hello.consumer
Bundle ID: 13
$ start 13
Starting bundle 13 (hello.consumer)...
Hello, World !
$ stop 12
Stopping bundle 12 (hello.provider)...
Hello, the end !
\end{minted}
\end{block}
\end{frame}
