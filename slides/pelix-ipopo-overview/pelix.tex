\begin{frame}{Pelix: Concepts}
\begin{block}{Services}
\begin{itemize}
\item Same concept as in OSGi:
\begin{itemize}
\item object implementing interfaces,
\item associated to properties
\end{itemize}
\item Pelix supports:
\begin{itemize}
\item Multiple interfaces services
\item Service ordering (\texttt{service.ranking})
\item Service filtering (with LDAP filters)
\end{itemize}
\item \texttt{ServiceRegistration} and its \texttt{ServiceReference}
\item Support for Service Factory, Service Prototype Factory
\end{itemize}
\end{block}
\end{frame}

\begin{frame}{Pelix: Concepts}
\begin{block}{Bundles}
\begin{itemize}
\item Bundle $\Leftrightarrow$ single Python module
\begin{itemize}
\item Standard \texttt{.py}/\texttt{.pyc}/\texttt{.pyo} file
\item Native module (\texttt{.pyd})
\end{itemize}
\item Support of the Bundle Activator
\begin{itemize}
\item Allows a life cycle for the module
\item Standard bundles can be updated at runtime
\end{itemize}
\item \texttt{Bundle} and its \texttt{BundleContext}
\end{itemize}
\end{block}
\end{frame}


\begin{frame}{Pelix Bundle Life Cycle}
\centering
\tikzset{
    %Define standard arrow tip
    >=stealth',
    %Define style for boxes
    punkt/.style={
           rectangle,
           rounded corners,
           draw=black, very thick,
           fill=yellow!30,
           text width=6.5em,
           minimum height=2em,
           text centered},
    % Define arrow style
    pil/.style={
           ->,
           thick,
           shorten <=2pt,
           shorten >=2pt,}
}

\begin{tikzpicture}[node distance=2.3em, auto]
\node[circle, draw, very thick, fill=yellow!30] (init) {};
\node[punkt, below=of init] (installed) {INSTALLED};
\node[punkt, below=of installed] (resolved) {RESOLVED};
\node[punkt, below=of resolved] (uninstalled) {UNINSTALLED};
\node[punkt, right=of installed] (starting) {STARTING};
\node[punkt, right=of resolved] (active) {ACTIVE};
\node[punkt, right=of uninstalled] (stopping) {STOPPING};
\node[circle, draw, very thick, fill=black, below=of uninstalled] (final) {};

\draw[pil] (init.south) to node[auto, swap] {Install} (installed.north);
\draw[pil] (installed.south) to node[auto] {} (resolved.north);
\draw[pil] (resolved.east) to node[auto] {Start} (starting.west);
\draw[pil] (starting.south) to node[auto] {} (active.north);
\draw[pil] (active.south) to node[auto] {Stop} (stopping.north);
\draw[pil] (stopping.west) to node[auto] {} (resolved.east);
\draw[pil] (resolved.south) to node[auto, swap] {Uninstall} (uninstalled.north);
%\draw[pil, bend right=35] (resolved.west) to node[auto] {} (uninstalled.west);
\draw[pil] (uninstalled.south) to node[auto] {} (final.north);
\end{tikzpicture}

\begin{exampleblock}{Note}
In Pelix, \texttt{INSTALLED} is just a step state.
\end{exampleblock}
\end{frame}


\subsection{Snippets}

\begin{frame}[fragile]{Pelix API: Framework life cycle}
\begin{block}{Create a framework}
\begin{minted}{python}
from pelix.framework import create_framework

framework = create_framework(
	['pelix.ipopo.core'],
	{"foo": "bar"})
framework.start()
\end{minted}
\end{block}

\begin{block}{Stop a framework}
\begin{minted}{python}
# Stop the framework
framework.stop()

# Delete it
framework.delete()
framework = None
\end{minted}
\end{block}
\end{frame}

\begin{frame}[fragile]{Pelix API: Services}
\begin{block}{Register a service}
\begin{minted}{python}
# Service object
service = MyImplementation()

# Register the service
# context is the BundleContext
svc_reg = context.register_service(
	"my.interface", service,
	{"answer": 42}
\end{minted}
\end{block}

\begin{block}{Unregister a service}
\begin{minted}{python}
svc_reg.unregister()
\end{minted}
\end{block}
\end{frame}

\begin{frame}[fragile]{Register a service (in Java)}
\begin{block}{Register a service}
\begin{minted}{java}
// Instantiate the service object
MyInterface service = new MyImplementation();

// Set up properties
Hashtable properties = new Hashtable();
properties.put("answer", 42);

// Register the service
ServiceRegistration<?> svcReg = context.registerService(
	MyInterface.class, service, properties);
\end{minted}
\end{block}

\begin{block}{Unregister a service}
\begin{minted}{java}
svcReg.unregister();
\end{minted}
\end{block}
\end{frame}
