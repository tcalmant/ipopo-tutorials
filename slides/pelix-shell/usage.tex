\begin{frame}[fragile]{How to run the Shell}
\begin{block}{Run a Text UI}
\begin{minted}{bash}
user@localhost $ python -m pelix.shell
** Pelix Shell prompt **
$
\end{minted}
\end{block}

\begin{block}{Run from code}
\begin{minted}{python}
# Start the Shell Core bundle
ctxt.install_bundle('pelix.shell.core').start()

# Install iPOPO commands
ctxt.install_bundle('pelix.shell.ipopo').start()

# Start the console UI
ctxt.install_bundle('pelix.shell.console').start()
\end{minted}
\end{block}
\end{frame}

\begin{frame}{Basic Commands}
\begin{block}{Survival Kit}
\centering
\begin{footnotesize}
\begin{tabular}{r p{.45\textwidth}}
\texttt{\scriptsize help [<command>]} & Shows the help \\
\hline
\texttt{\scriptsize loglevel [<level>] [<name>]} & Prints/changes the log level \\
\hline
\texttt{\scriptsize exit} & Quits the shell \newline (the framework in console UI) \\
\hline
\texttt{\scriptsize threads} & Prints the stack trace of all threads \\
\hline
\texttt{\scriptsize run <file>} & Executes the given shell script \\
\end{tabular}
\end{footnotesize}
\end{block}

\begin{block}{Session Commands}
\centering
\begin{footnotesize}
\begin{tabular}{r l}
\texttt{set {\scriptsize [<property=value>]}} & Prints/sets session variables \\
\hline
\texttt{unset {\scriptsize <name>}} & Deletes a session variable \\
\end{tabular}
\end{footnotesize}
\end{block}

\begin{block}{Properties Commands}
\centering
\begin{footnotesize}
\begin{tabular}{r l}
\texttt{properties} & Lists the properties of the framework \\
\hline
\texttt{sysprops} & Lists the environment variables \\
\end{tabular}
\end{footnotesize}
\end{block}
\end{frame}


\begin{frame}{Basic Commands}
\begin{small}
\begin{block}{Bundle Commands}
\centering
\begin{tabular}{r l}
\texttt{install {\scriptsize <name>}} & Installs a module as a bundle \\
\hline
\texttt{start {\scriptsize <bundle ID>}} & Starts the given bundle \\
\hline
\texttt{update {\scriptsize <bundle ID>}} & Updates the given bundle \\
\hline
\texttt{uninstall {\scriptsize <bundle ID>}} & Uninstalls the given bundle \\
\hline
\texttt{bl {\scriptsize [<name>]}} & Lists the installed bundles \\
\hline
\texttt{bd {\scriptsize <bundle ID>}} & Prints the details of a bundle \\
\end{tabular}
\end{block}

\begin{block}{Service Commands}
\centering
\begin{tabular}{r l}
\texttt{sl {\scriptsize [<specification>]}} & Lists the registered services \\
\hline
\texttt{sd {\scriptsize <service ID>}} & Prints the details of a service \\
\end{tabular}
\end{block}
\end{small}
\end{frame}


\begin{frame}{iPOPO Commands}
\begin{block}{Information Commands}
\centering
\begin{small}
\begin{tabular}{r l}
\texttt{\scriptsize factories [<name>]} & Lists registered component factories \\
\hline
\texttt{\scriptsize factory <name>} & Prints the details of a factory \\
\hline
\texttt{\scriptsize instances [<name>]} & Lists components instances \\
\hline
\texttt{\scriptsize instance <name>} & Prints the details of a component \\
\hline
\texttt{\scriptsize waiting [<name>]} & Lists the components waiting for an handler \\
\end{tabular}
\end{small}
\end{block}

\begin{block}{Instance Commands}
\centering
\begin{footnotesize}
\begin{tabular}{p{.45\textwidth} l}
\texttt{\scriptsize instantiate <factory> <name> [<property=value>]} & Prints the registered services \\
\hline
\texttt{\scriptsize kill <name>} & Kills a component \\
\hline
\texttt{\scriptsize retry <name> [<property=value>]} & Retry the validation of a component \\
\end{tabular}
\end{footnotesize}
\end{block}
\end{frame}

\begin{frame}{Shell Ecosystem}
\begin{small}
\begin{exampleblock}{Note}
All these modules are in the \texttt{pelix.shell} package.
\end{exampleblock}

\begin{block}{Existing UIs}
\centering
\begin{tabular}{rl}
\texttt{console} & Console (text) UI \\
\hline
\texttt{remote} & Access the shell via TCP, e.g. via NetCat \\
\hline
\texttt{xmpp} & Access the Shell via XMPP (based on SliXMPP) \\
\end{tabular}
\end{block}

\begin{block}{Command providers}
\centering
\begin{tabular}{rl}
\texttt{configadmin} & Commands to create and update configurations \\
\hline
\texttt{eventadmin} & Commands to send/post events \\
\hline
\texttt{ipopo} & Commands to manage iPOPO instances \\
\hline
\texttt{report} & Generates a JSON report about the framework \\
\end{tabular}
\end{block}
\end{small}
\end{frame}
